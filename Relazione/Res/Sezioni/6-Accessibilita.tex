\section{Accessibilità}
	\subsection{Schema colori}
	Si è cercato di utilizzare uno schema colori tale che garantisca un contrasto elevato, in modo da facilitare la lettura del contenuto anche alle persone con disturbi visivi. Inoltre, i link vengono sempre rappresentati sottolineati e rispettando sempre la convenzione interna del sito. Per garantire che il sito sia accessibile anche alle persone con disturbi visivi è stato utilizzato il servizio offerto dal sito \url{http://www.color-blindness.com/coblis-color-blindness-simulator/} che a partire da uno screenshot di una pagina, mostra come viene visualizzata da persone con determinati disturbi visivi. Viene di seguito riportato il risultato ottenuto dal test della homepage.

	\subsection{Utilizzo dei tag}
	Sono stati inseriti per ogni pagina i tag \textit{meta}: \textit{Content-Type}, \textit{keywords}, \textit{description}, \textit{author} e \textit{languages} e il tag \textit{title}, il quale descrive la pagina corrente \textbf{dal particolare al generale}. Il tag \textit{languages} indica che il sito è stato interamente scritto in italiano ma compaiono alcune parole inglesi, le quali sono state affiancate dall’attributo: \textit{xml:lang=”en”}.
	
	\subsection{Screen reader}
	Ogni locandina è stata arricchita di attributi \textit{alt} e \textit{title} che descrivono in maniera esaustiva ciò che l'immagine ritrae. Per le immagini che sono state ritenute non di contenuto, e che sono così state inserite tramite CSS, non è stato previsto l'uso di questi attributi, poiché la loro unica funzione è di presentazione, non portando valore aggiunto al contenuto.
	Ogni campo di un form è stato sempre corredato con una etichetta \textit{label}.
	
	\subsection{Aiuti per la navigazione}
	Al fine di aumentare l'accessibilità del sito sono previsti le seguenti facilitazioni per la navigazione:
	\begin{itemize}
		\item \textbf{Colori dei link}: per evitare il disorientamento durante la navigazione, sono previsti colori differenti per i link attivi visitati (\textbf{blue}) e quelli non visitati (\textbf{viola});
		\item \textbf{Tabindex}: ad ogni pressione del tasto tab il focus si sposta sul link direttamente successivo per agevolare la navigazione. Sono stati ridefiniti gli attributi \textit{tabIndex} dei link in modo da rispecchiare l'ordine desiderato;
		\item \textbf{Link per spostarsi al contenuto}: prima del menù di navigazione è stato inserito un link, nascosto all'utenza normale, ma che permette agli utenti che visualizzano il sito mediante uno screen reader di saltare la barra di navigazione;
		\item \textbf{Link per tornare al menù}: Per facilitare la navigazione nel sito sono stati creati dei link interni alla pagina che permettono agli utenti di tornare al menù di navigazione.
	\end{itemize}
	


