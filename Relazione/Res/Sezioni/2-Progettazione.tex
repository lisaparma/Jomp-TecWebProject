\section{Progettazione}
	\subsection{Database}
Il sito, per funzionare correttamente, è connesso ad un database costruito in linguaggio SQL. È stato scelto questo linguaggio in quanto tutti i componenti del gruppo lo conoscono e ne hanno dimestichezza. 
All'interno del database vi sono diverse tabelle: le principali sono Utenti, Aziende e Annunci, e ognuna di esse si appoggia ad altre tabelle minori come Consultazioni, Tipo, OrarioLavoro, ContrattoLavoro. Le ultime tre, in particolare, raccolgono la tipologia di lavoro, orario e contratto che un impiego può avere e si è scelto di inserirle in una tabella per evitare che l'utente possa modificare la voce nel menù a tendina.   

\subsection{Gerarchia dei file}
I file che contengono il sito sono organizzate in 4 cartelle: 
\begin{itemize}
	\item \textbf{css}: contiene il foglio di stile a cui fanno riferimento le pagine .php;
	\item \textbf{IMG}: contiene le immagini presenti nel sito;
	\item \textbf{JavaScript}: contiene i vari script realizzati in JavaScript;
	\item \textbf{php}: contiene i file .php che generano le pagine del sito con codice html;
\end{itemize}
Sono inoltre presenti il file database.sql, utilizzato per creare il database del sito, e popolamento.txt, che contiene i dati di prova che popolano il database.


\subsection{Test}
Il corretto funzionamento del sito è stato testato sui browser:
\begin{itemize}
	\item Safari 11.0.3
	\item Chrome 64.0.3282.140
	\item Edge 	41.00
	\item Internet Explorer 9
\end{itemize}
E su i dispositivi mobili:
\begin{itemize}
	\item iPhone 7
	\item iPhone 5
	\item Huawaii Honor
	\item iPad mini 5
	\item Samsung Galaxy Tab
\end{itemize}