\section{Presentazione}
	La presentazione è stata fatta in CSS3 tenendola completamente separata dalla struttura attraverso due file esterni, uno per gestire la visualizzazione in un qualsiasi schermo e un'altro per gestire la stampa.
	\subsection{Scelta CCS 3}
	È stato deciso di utilizzare l'ultimo standard di CSS dato che, essendo uscito nel 2008, tutte le sue caratteristiche aggiuntive rispetto al CSS2 sono ben supportate in tutti i browser. In particolare le novità che sono state utili sono:
	\begin{itemize}
		\item \textbf{Nuove pseudo-classi}: \emph{:first-child}, \emph{:last-child}, e \emph{:nth-child(n)} sono state utilizzate per gestire il breadcrub;
		\item \textbf{Nuove proprietà}: sono state largamente utilizzate la nuove proprietà \emph{border-radius} e \emph{box-shadow};
		\item \textbf{Transizioni}: sono state modificate delle transizioni con \emph{transition-property} e \emph{transition-duration} per rendere meno forte il cambio di visualizzazione per l'utente;
	\end{itemize}
	\subsection{desktop.css}

	\subsection{print.css}
	Questo foglio di stile si applica automaticamente quando un utente vuole
stampare la pagina. Sono stati tolti tutti gli elementi visivi non strettamente necessari, quindi tutti i colori e le immagini di background o di presentazione; abbiamo mantenuto solo i bordi dei form colorati. Abbiamo rimosso anche il menu principale dall'header e il menu laterale delle aree personali in quanto non ne abbiamo ritenuta fondamentale la visualizzazione su una pagina stampata. Una caratteristica di questo file è la presenza di un forte raggruppamento dei tag che utilizzano la stessa regola. La parte iniziale sarà quindi un elenco di tag seguiti dalla regola che li accomuna. Questa decisione peggiora la leggibilità dei fogli di stile, ma ne alleggerisce notevolmente il peso.




