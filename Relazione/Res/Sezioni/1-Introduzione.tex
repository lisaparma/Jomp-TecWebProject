\section{Introduzione}
	\subsection{Abstract}
	Il progetto Jomp si propone di implementare un sito internet per facilitare la ricerca di lavoro da parte degli utenti e la ricerca di personale da parte delle aziende. Il sito dispone di una semplice e intuitiva barra di ricerca in cui si possono filtrare le proposte di lavoro per titolo dell'annuncio, luogo di lavoro e tipologia di impiego, dando luogo all'elenco di annunci corrispondenti.
	Il sito è stato sviluppato rispettando gli standard W3C, la separazione tra struttura, presentazione, comportamento e le regole di accessibilità richieste. 
	\subsection{Utenti destinatari}
	Come menzionato precedentemente, il sito è rivolto a chi cerca o offre lavoro, ragion per cui il target è definito da soggetti adulti (la registrazione può avvenire solo se si è maggiorenni).
Al sito possono accedere diverse categorie di utenti: 
\begin{itemize}
\item gli utenti registrati, che possono salvare gli annunci per loro più interessanti e vederne i dettagli (per esempio, orario di lavoro e tipo di contratto proposto); 
\item le aziende registrate, che possono pubblicare i loro annunci di lavoro o modificare quelli precedentemente inseriti;
\item gli utenti non registrati, che posso ricercare annunci senza però poterli salvare o vedere i dettagli dell'offerta di lavoro (le aziende non registrate non possono aggiungere annunci di lavoro); 
\item l'amministratore, che gestisce il sito e può rimuovere utenti, aziende e annunci.
\end{itemize}
